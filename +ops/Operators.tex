\documentclass{article}
\usepackage{amsmath,amsfonts,amssymb,amsthm}
\usepackage{mathrsfs}
\usepackage{graphicx}
\usepackage{color}

\sloppy
\definecolor{lightgray}{gray}{0.5}
\setlength{\parindent}{0pt}

\title{Class for boundary operators}

\begin{document}
\maketitle    
\def\Po{{ $\mathbb{P}_1$ }}   
\def\Pz{{ $\mathbb{P}_0$ }}   

\section{Introduction}
Let $D_1, D_2$ be two $C^2$ shapes (C2boundary objects) which are either disjoint, or identical. Define $A: L^2(\partial D_1) \rightarrow L^2(\partial D_2)$ a linear operator which has the form:
$$Af(y) = \int_0^{2\pi} k(y, x(t)) f(x(t)) |x'(t)| dt, \ y\in\partial
D_2 $$ where $k(y,x)$ is the integration kernel. We define the \emph{kernel matrix}:
$$(K)_{ji}:=k(y(t_j), x(t_i))\sigma_1(t_i),$$
with $\sigma_1$ the vector of integration elements of $\partial D_1$, and in case that $A$ is identity, we define $K_{ji}=\delta_{ji}$. Then $Af(y(t_j))$ can be approximated by:
$$\sum_i k(y(t_j), x(t_i)) f(x(t_i)) \sigma_1(t_i) = \sum_i K_{ji}f(x(t_i)).$$

\subsection{Stiffness matrix}
Consider the linear system:
$$Af = b, \ f\in L^2(\partial D_1), \mbox{ and } b\in L^2(\partial D_2)$$
We solve it in an internal approximation space of $L^2(\partial D_1)$ generated by a boundary element basis $\{\psi_n, n=1..N\}$. The corresponding variational formulation reads:
$$\sum_n \langle A\psi_n, \phi_m \rangle f_n = \langle b, \phi_m\rangle, \mbox{ for } m=1..M$$
with $\{\phi_m, m=1..M\}$ a boundary element basis of $\partial D_2$.  We introduce the \emph{stiffness matrix} $(A)_{mn}:=\langle A\psi_n, \phi_m \rangle$, where $\psi_n, \phi_m$ are basis functions. In the current version, the class \texttt{Operators} can handle the \Pz and the \Po basis. The \Pz basis is defined in a discrete manner as $\psi_m(y(t_j))=\delta_{mj}$ (similarly for $\phi_n$), and the \Po basis is the classical hat function. The scalar product as well as the stiffness matrix $A$ are computed differently depending on the combination of $\psi_n, \phi_m$.

If $\phi_m$ is the \Pz basis, then the scalar product $\langle A\psi_n, \phi_m \rangle$ is understood as the point value of $A\psi_n$ at the boundary point $x(t_m)\in\partial D_2$, so the stiffness matrix is approximated by:
$$(A)_{mn} \simeq \sum_j\phi_m(y(t_j))\sum_i K_{ji} \psi_n(x(t_i)).$$ 
If $\phi_m$ is the \Po basis, the scalar product is understood as a boundary integral on $\partial D_2$, and the stiffness matrix is approximated by:
$$(A)_{mn} \simeq \sum_j\phi_m(y(t_j))\sigma_2(t_j)\sum_i K_{ji} \psi_n(x(t_i)).$$

By Px-Py we mean taking the Px basis for $\psi_n$ and Py basis for
$\phi_m$, then one has the following expressions for $(A)_{mn}$: 
\begin{itemize} 
\item \Pz-\Pz: $K_{mn}$ 
\item \Pz-\Po: $\sum_j \phi_m(y(t_j)) \sigma_2(t_j) K_{jn} =(\Phi^\top\Sigma_2 K)_{mn}$
\item \Po-\Pz: $\sum_i K_{mi}\psi_n(x(t_i)) = (K\Psi)_{mn}$ 
\item \Po-\Po: $\sum_j\phi_m(y(t_j))\sigma_2(t_j)\sum_i K_{ji} \psi_n(x(t_i)) = (\Phi^\top\Sigma_2 K\Psi)_{mn}$
\end{itemize} 
Here $\Sigma_2=\mbox{diag}(\sigma_2(t_1)\ldots)$, $\Phi=[\phi_1, \ldots \phi_M]$, and $\Psi=[\psi_1,\ldots \psi_N]$.

\end{document}
    
